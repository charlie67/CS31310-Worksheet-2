\documentclass[10pt]{article}
\usepackage{geometry}                % See geometry.pdf to learn the layout options. There are lots.
\geometry{a4paper
}                   % ... or a4paper or a5paper or ... 
\usepackage[parfill]{parskip}    % Activate to begin paragraphs with an empty line rather than an indent

%%%%%%%%%%%%%%%%%%%%
\newcommand{\hide}[1]{}

\usepackage{natbib}
\usepackage{xcolor}
\usepackage{url}
\usepackage{hyperref}
\usepackage{mathtools}
\usepackage{array}
\usepackage{multirow}
\usepackage{graphicx}
\usepackage{subcaption}
\usepackage{mwe}
\usepackage{float}
\usepackage{tikz}

\hide{
\usepackage{amscd}
\usepackage{amsfonts}
\usepackage{amsmath}
\usepackage{amssymb}
\usepackage{amsthm}
\usepackage{cases}		 
\usepackage{cutwin}
\usepackage{enumerate}
\usepackage{epstopdf}
\usepackage{graphicx}
\usepackage{ifthen}
\usepackage{lipsum}
\usepackage{mathrsfs}	
\usepackage{multimedia}
\usepackage{wrapfig}
}

\title {CS31310 Agile Assignment}
\begin{document}
\maketitle

“Simplicity --the art of maximizing the amount of work not done-- is essential.”

This is significant in delivering valuable software because if you want to quickly and simply write software then you don't want to be doing work that didn't need doing. When writing software you want to create the simplest and easiest thing that follows the customers requirements and you're not doing that if you're writing extra code and creating extra work for yourself.

This relates strongly to "Responding to change over following a plan" from the agile manifesto because if you want to be quick to respond to the changing needs of software development then you don't want to have to re-engineer code that didn't need to be written in the first place.

This also relates to the agile principle "Our highest priority is to satisfy the customer
through early and continuous delivery of valuable software." because if you are doing work that doesn't need to be done then you're going to be slower at delivering the software to the customer and the customer is not going to be as happy because of this increased delivery time.

"Welcome changing requirements, even late in development. Agile processes harness change for the customer's competitive advantage.". The statement relates to this principle because when the requirements change this can sometimes result in a lot of new work and a lot of code that needs to be changed and if you've been writing extra code that didn't need to be written then you're going to have a harder time when requirements change. During my industrial year the company I was working for was following an agile approach and was therefore continuously delivering software to the customer. After one delivery quite late in the project the customer came back with some comments saying how they wanted a particular feature to work in a different way, this obviously resulted in some refactoring and some changes but because the code was quite separated and modular and they were writing software simply it wasn't as bad as it could of been. It also resulted in a significantly improved experience for the customer because they quickly got a new version of the software with their requested changes in.

"Deliver working software frequently, from a couple of weeks to a couple of months, with a preference to the shorter timescale." This is the third agile principle and this relates to the statement because you want to deliver to the customer quickly and frequently, this is going to be harder to do if you're spending longer writing complicated code that didn't need to be written then the customer will have to wait longer for working software.

"Working software is the primary measure of progress.". If you're over complicating software then you will be slower to make working software which by this measure means that you're not progressing as quickly as you could be. If you simplified your work and don't write code that doesn't need to exist then you're going to progress with your tasks and produce working software quicker. This also ties into continuously delivering software to the customer, if you're progressing quicker by writing your code in a simple way and not doing work that doesn't need to be done then you can deliver software quicker and satisfy the customer.

"Continuous attention to technical excellence and good design enhances agility.". If you're always paying attention to the technical, architectural and design side of your software creation then you will quickly notice work that shouldn't be happening and you can be in a position to stop it before it spirals and lots of time has been wasted on work that didn't need to happen. 

"At regular intervals, the team reflects on how to become more effective, then tunes and adjusts its behavior accordingly.". During my industrial year we had these periodically and called them retrospectives, these were designed to allow the team time to talk and discuss what was going well in the process and what could be improved. During retrospectives would be the time to talk about work that was happening that was maybe not essential and could be cancelled or delayed slightly. This team communication is vital to ensure that everyone knows what's going on and can voice their opinions on the process.

Whilst I was working on a project during my industrial year we had to start writing code for a project before we had concrete requirements from the customer, this did obviously result in work that had to be scrapped or re-worked when the customer clarified their position on certain features. This was not an agile approach and did not follow the principle “Simplicity --the art of maximizing the amount of work not done-- is essential.”. However at that time it was necessary to do this due to time constraints. We did mitigate the re-work required by doing the minimal amount of work required and writing code in a simple way to ensure that we did maximise the amount of work that we didn't need to do.

\end{document}  
%%%%%%%%%%%%%%%%%%%%%%%%%%%%%%%%%%%%%%%%%%%%%%